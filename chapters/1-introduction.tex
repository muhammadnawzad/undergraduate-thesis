\pagenumbering{arabic}

\chapter{Introduction}

\begin{justify}
The Student Talent Development Center (STDC) at the University of Kurdistan Hewlêr plays a crucial role in supporting the academic and personal development of students through various services, such as Peer Assisted Learning (PAL). \cite{topping1998peer} define Peer Assisted Learning (PAL) as "people from similar social groups who are not professional teachers helping each other to learn and learning themselves by teaching". The STDC is able to encourage the development of student skills and abilities by creating a community of learners who work together toward common goals.
\vspace{0.25cm}
\newendline In order to enhance the efficiency and effectiveness of its operations, the STDC is seeking to implement a scheduling, feedback, and resource management system that can streamline its workflow. Course timetabling is a complex problem that involves assigning students, teachers, courses, and classrooms to specific times and locations \cite{carter1998recent}. Many universities have turned to virtual learning platforms, such as Moodle, to facilitate this process \cite{al2008moodle}. However, these platforms may not always meet the specific needs and requirements of a given institution. As \cite{alvarez2002design} note, the format of schedules can vary significantly from one university to another, and it is important to consider the particular constraints and objectives of the course scheduling process.
\vspace{0.25cm}
\newendline Course Scheduling, Class-Teacher Scheduling, Student Scheduling, Teacher Assignment, and Classroom Assignment are all few of the subproblems described by \cite{carter1998recent} in their assessment. It is necessary to consider a variety of constraints and needs in order to solve any of these smaller problems. The goal of this project is to develop a web-based software system that can address these challenges and optimize the STDC's workflow. The system will allow tutors, students, and administrators to schedule and confirm tutoring sessions more efficiently, and provide a more comprehensive view of each group's progress to the administration.
\vspace{0.4cm}
To achieve this goal, it will be necessary to carefully design and implement the system, taking into account the specific needs and requirements of the STDC. This requires a thorough analysis of the current workflow and the identification of any bottlenecks or inefficiencies. It also involves the development of a database to store and manage the necessary data, as well as the creation of user-friendly interfaces for the various stakeholders.
\vspace{0.25cm}
\newendline The proposed system for STDC enhances the efficiency and effectiveness of the center’s duties. By automating routine tasks and providing a more comprehensive view of each group's progress, the system helps to optimize the use of resources and support the development of student talent.
\vspace{0.25cm}
\newendline In addition to the benefits for the STDC, the implementation of a scheduling, feedback, and resource management system has the potential to contribute to the broader field of education. The demand for efficient and effective solutions to aid in the administration and distribution of education is growing as the pace of technological development accelerates. The development of a web-based software solution to handle the issues that a center like STDC faces has the potential to serve as a model for other institutions looking to better their workflows.\\
\end{justify}

\section{Problem Statement}
\begin{justify}
Scheduling tutoring sessions for students, maintaining records on tutor availability and attendance, gathering feedback from students on tutors, allocating rooms, and making available tutoring modules visible are all regular parts of the Student Talent Development Center's routine. Beginning with students, they have trouble accessing the provided material in each session, knowing how many modules are available to them, giving feedback on certain sessions, and when and where each session is being held. Then, tutors are unable to indicate their availability for upcoming sessions to students, see the total number of recorded tutoring hours, or monitor session attendance. Consequently, the administration has no way of knowing how many sessions a tutor has had with a particular student. They have trouble allocating rooms for each session, tracking how much each tutor should be paid, and receiving useful feedback from students.\\
\end{justify}

\section{Objectives}
\begin{justify}
    \begin{itemize}
        \item Automation of UKH Student Talent Development Center’s Manual Workflow.\\
    \end{itemize}
\end{justify}

\section{Thesis Organisation}
\begin{justify}
After Chapter 1 Introduction, which provides an overview of the project and its context, Chapter 2 Background and Literature Review presents a review of relevant literature and research on the topic. Next, Chapter 3 Methodology describes the methodology in which way the project has been developed.
\vspace{0.25cm}
\newendline Then, Chapters 4 Analysis and Design covers the planning phase, analysis phase, and design phase of the SDLC with each as section. These sections contain detailed steps taken to plan, analyze, and design the web-based software system. After that, Chapter 5 Implementation and Testing covers the implementation and the testing phase of SDLC including any challenges or issues that were encountered and how they were addressed. The testing phase of the SDLC is covered in the last section of chapter 5, which describes the various testing methods and techniques used to ensure the quality and reliability of the system.
\vspace{0.25cm}
\newendline Finally, Chapter 6 Conclusion summarizes the main findings of the project and discusses the contributions of the study. It also provides recommendations for future work and potential areas for further research. Overall, the structure of the thesis reflects a systematic and rigorous approach to the research process, following a logical progression from the introduction to the conclusion.
\end{justify}

\clearpage
