\chapter{Background and Literature Review}

\begin{justify}
Virtual learning platforms, such as Moodle, Google Classroom, and Edmodo, have become increasingly popular in higher education as a means of supporting online and blended learning. These platforms offer a range of tools and features for communication, collaboration, and assessment, and are designed to facilitate the delivery of course content and support student learning.

\vspace{0.25cm}
\newendline
Several studies have explored the use and effectiveness of virtual learning platforms in higher education, some of these studies will be mentioned briefly while others will be reviewed thoroughly in the following review.

\vspace{0.25cm}
\newendline The use of intelligent tutoring systems has been studied way back in 1985 by \cite{anderson1985intelligent}, who examined the use of intelligent tutoring systems in education, noting their potential to provide personalized and adaptive instruction. Then, \cite{vanlehn2006behavior} studied the behavior of tutoring systems, identifying factors that can influence their effectiveness, such as the type of task and the level of student engagement. Furthermore, \cite{vanlehn2011relative} compared the relative effectiveness of human tutoring, intelligent tutoring systems, and other tutoring systems, finding that all three types of tutoring can be effective in supporting student learning.

\vspace{0.25cm}
\newendline Additionally, \cite{aastrom2008feedback} examined feedback systems, highlighting the importance of timely and specific feedback in supporting learning. Then, \cite{tadelis2016reputation} explored the role of reputation and feedback systems in online platform markets, noting their potential to increase trust and encourage participation. Furthermore, \cite{sadik2017students} studied the acceptance of file-sharing systems, such as Google Drive, by students as a tool for sharing course materials.

\vspace{0.25cm}
\newendline There are several other types of applications that can be seen between 2008 – 2010 including the \cite{zhou2010lesson}’s system that was developed for lecture notes searching and sharing, called LESSON, which allows users to easily locate and access notes over the internet. Then, \cite{wang2010peerlearning} proposed PeerLearning, a content-based e-learning material-sharing system based on a peer-to-peer network. In addition to these, \cite{lan2007mobile} examined the use of a mobile-device-supported PAL system for collaborative early English as a foreign language reading.

\vspace{0.25cm}
\newendline Many studies have been done on the many kinds of virtual learning platforms, which may be seen to come in a variety of flavors. These platforms are the subject of a great deal of research. However, in order to determine which of these platforms have the most significant impact and importance to this study, it is necessary to conduct an analysis of platforms such as Moodle, Google Classroom, and Edmodo. This is due to the fact that, among the various virtual learning platforms, these three platforms have come the closest to satisfying the requirements of the Student Talent Development Center.

\vspace{0.25cm}
\newendline Starting with Moodle which is a free, open-source learning management system (LMS) that is widely used by institutions around the world. According to \cite{rice2006moodle}, Moodle is built on top of a LAMP stack and uses a modular design, which allows users to customize the platform and add additional features through the use of plugins. This flexibility and customizability is a major advantage of Moodle, as institutions can tailor the platform to meet their specific needs and requirements. However, setting up and maintaining Moodle may require a significant amount of time and resources, particularly for larger institutions. \cite{dougiamas2004moodle} notes that Moodle is generally considered to be user-friendly and reliable, with a range of tools and features for communication, collaboration, and assessment.

\vspace{0.25cm}
\newendline \cite{kc2017evaluation} conducted a case study of the Moodle-based course management system at Kajaani University of Applied Sciences, finding that the platform was well-received by students and faculty, with a high level of satisfaction reported. However, the study also identified several challenges, including technical issues and a lack of training for faculty. Similarly, \cite{al2008moodle} evaluated the Moodle-based course management system at King Saud University and found that while the platform was well-received, there were also challenges such as a lack of training and technical difficulties.

\vspace{0.25cm}
\newendline In terms of efficiency, Moodle offers a range of tools and features that can help institutions streamline their workflows and improve the delivery of course content. For example, Moodle allows instructors to create and manage online courses, schedule lectures, upload resources and content such as lecture files, and provide feedback to students. It also enables students to access course materials, submit assignments, and collaborate with their peers. Overall, Moodle is a powerful and widely-used LMS that can help institutions support online and blended learning, although its effectiveness may depend on its implementation and the specific needs of the institution.

\vspace{0.25cm}
\newendline In contrast with Moodle, Google Classroom is also a free, web-based learning management system (LMS) developed by Google that is built on top of the Google Drive and Google Calendar platforms. According to \cite{iftakhar2016google}, Google Classroom is designed to be user-friendly and easy to use, with a simple and intuitive interface that is familiar to many users. \cite{sudarsana2019use} found that the use of Google Classroom in the learning process was beneficial, with students reporting improved organization and communication. However, \cite{mohd2016application} note that there may be limitations to the customizability of Google Classroom, as it is built on top of a specific set of tools and features that cannot be easily modified.

\vspace{0.25cm}
\newendline \cite{tadelis2016reputation} discusses the role of reputation and feedback systems in online platform markets, stating that they can be effective in improving the quality and efficiency of services. Google Classroom includes a range of tools and features for communication, collaboration, and assessment, and is highly scalable, integrating with other Google tools such as Gmail, Docs, and Sheets. \cite{sadik2017students} found that students had a high level of acceptance of file-sharing systems like Google Drive as a tool for sharing course materials. However, it is important for institutions to consider issues of data privacy and security when using cloud-based LMSs like Google Classroom.

\vspace{0.25cm}
\newendline In terms of efficiency, Google Classroom is generally considered to be user-friendly and reliable, with a range of tools and features for communication, collaboration, and assessment. However, as noted by \cite{mohd2016application}, the effectiveness of the platform may depend on the specific design and implementation of the program, as well as the characteristics of the students and instructors involved. Overall, Google Classroom is a popular LMS that offers a range of benefits for online and blended learning, but it is important for institutions to carefully consider their specific needs and requirements when evaluating its use.

\vspace{0.25cm}
\newendline Just like both of the above platforms, Edmodo is also a cloud-based learning management system (LMS) that is designed to be user-friendly and easy to use, with a simple and intuitive interface. According to \cite{dewi2014edmodo}, Edmodo is a social learning platform that can be used for blended learning in higher education. The platform includes a range of tools and features for communication, collaboration, and assessment, and is focused on education and the integration of various educational tools and resources. Edmodo is highly scalable and has a strong emphasis on security and data privacy per \cite{gay2017effectiveness}. However, it may not be as customizable as some other LMSs, and some users may find it less feature-rich compared to other platforms such as Moodle or Google Classroom.

\vspace{0.25cm}
\newendline \cite{gay2017effectiveness} conducted a study to evaluate the effectiveness of using Edmodo in enhancing student outcomes in an advanced writing course at FIP-UMMU. The authors found that Edmodo was well-received by students and facilitated collaborative learning. However, they also identified some challenges, including a lack of technical support and a need for more training for faculty.

\vspace{0.25cm}
\newendline In terms of scheduling, Edmodo includes a calendar feature that allows users to schedule events, assignments, and assessments. Edmodo cannot be used to schedule lectures or other large-scale events. However, the platform does allow users to upload and share resources such as lecture files, and it includes features for providing feedback on lectures and other course materials. Overall, Edmodo is a user-friendly and reliable LMS that can support communication, collaboration, and assessment in higher education.

\vspace{0.25cm}
\newendline The above review of the literature on virtual learning platforms in higher education reveals that several studies have explored the use and effectiveness of platforms such as Moodle, Google Classroom, and Edmodo. However, there is a lack of research and project on the use of web applications specifically for Tutoring Centers such as the University of Kurdistan Hewler’s Student Talent Development Center in the Kurdistan Region of Iraq (KRI).

\vspace{0.25cm}
\newendline Additionally, much of the existing research has focused on the use of these platforms for course management and delivery for universities, rather than specifically for a tutoring center. This suggests a need for research or a project on the effectiveness of web applications for supporting and enhancing tutoring centers in KRI.

\vspace{0.25cm}
\newendline Another reason to consider the development of a new web application is the potential for increased efficiency and effectiveness. A custom web application could be designed to specifically meet the needs and requirements of the university's talent development center, potentially leading to a more streamlined and efficient process. Additionally, a custom web application could offer a range of features and tools that are not available on existing platforms, such as session registration, real-time feedback, and specialized resources and materials.

\vspace{0.25cm}
\newendline In summary, the proposed system aims to automate the manual workflow of the tutoring center. By implementing a student talent development center web application for the University of Kurdistan Hewlêr, the project will be the first of its kind in the Kurdistan Region of Iraq (KRI) and it will answer the question of whether such a web application is necessary and, if so, how it can be designed and implemented effectively.

\end{justify}
\clearpage